\documentclass{apecon}
\usepackage[T1]{fontenc}
\usepackage[latin1]{inputenc}
\usepackage{csquotes}
\MakeInnerQuote{"}

\usepackage{textcomp}
\usepackage{multido}

\usepackage{hyperref}
\hypersetup{%
   colorlinks = {true},
   urlcolor = {blue},
   linkcolor = {black},
   citecolor = {black},
   pdfauthor = {Arne Henningsen},
   pdftitle = {Testing LaTeX class and BibTeX style for the
      journal Applied Economics (ApEcon)},
   pdfkeywords = {Applied Economics, BibTeX, LaTeX}
}

\usepackage{multido}

\title{Testing \LaTeX{} class and Bib\TeX{} style for the
   journal `Applied Economics' (ApEcon)}

\author{Arne Henningsen$^1$ and His Coauthor$^2$}

% affiliation(s) and full address(es) of the author(s)
\affiliation{%
$^1$Institute of Food and Resource Economics,
University of Copenhagen,
Rolighedsvej~25, 1958~Frederiksberg~C, Denmark\\
$^2$\LaTeX{} University, Nostreet~123,
00000~Nowhere City, Noland}

% running title
\rtitle{\LaTeX{} class and Bib\TeX{} style for `Applied Economics'}

% corresponding author including full address
\cauthor{Arne Henningsen, Institute of Food and Resource Economics,
Rolighedsvej~25, 1958~Frederiksberg~C, Denmark}


\begin{document}

\maketitle

\begin{abstract}
\multido{}{15}{This is an abstract. }
\end{abstract}

\section{Introduction}
\multido{}{7}{This is an introduction. }

Footnotes should appear at the end of the page in which they are inserted.%
\footnote{
\multido{}{10}{This is a footnote. }
}
"Single quotation marks" can be conveniently inserted using
the "csquotes" package:
add the lines\\
\verb!\usepackage{csquotes}!\\
\verb!\MakeInnerQuote{"}!\\
to the preamble of your \LaTeX{} file and use the inch symbol~(\verb!"!)
for quotation marks.%
\footnote{%
Of course, you can also define another symbol in the command
\texttt{\textbackslash{}MakeInnerQuote},
e.g.\ the degree sign~($^{\circ}$).
}
Collect tables and figures at the end of the manuscript
(see figure~\ref{fig:dummy} and table~\ref{tab:citations}).

\begin{figure}[htbp]
\fbox{\parbox{0.6 \textwidth}{\centering
   \vspace{0.2 \textwidth}
   This is not a figure.
   \vspace{0.2 \textwidth}
}}
\caption{Dummy figure}
\label{fig:dummy}
\end{figure}

\begin{figure}[htbp]
\fbox{\parbox{0.6 \textwidth}{\centering
   \vspace{0.2 \textwidth}
   This is not a figure, too.
   \vspace{0.2 \textwidth}
}}
\caption{Figure with \multido{}{40}{very } long title}
\label{fig:long-title}
\end{figure}

\section{Manuscript Formatting}
Instructions to authors including formatting guidelines are available at
\url{http://www.tandf.co.uk/journals/journal.asp?issn=0003-6846&linktype=44}.
All references used as examples in these guidelines are shown in this document
to demonstrate that the Bib\TeX{} style of "Applied Economics" complies
with these guidelines.
Please report any problems at
\url{http://sourceforge.net/projects/economtex/}.


\section{Citations}
\subsection{Citations in Text}
\citet{smith72} says A, \citet{brown05} say B,
\citet{smith72a} say C, and \citet{smith72b} say D.
An overview is available in table~\ref{tab:citations}.

\begin{table}[htbp]
\caption{Citations}
\label{tab:citations}
\begin{tabular}{lc}
\hline
Author(s) & Statement\\
\hline
\citet{smith72} & A\\
\citet{brown05} & B\\
\citet{smith72a} & C\\
\citet{smith72b} & D\\
\hline
\end{tabular}
\end{table}


\subsection{Citations in Parenthesis}
A equals B \citep{smith72}, B equals C \citep{brown05},
C equals D \citep{smith72a}, and D equals A \citep{smith72b}.
Hence, A, B, C, and D are all equal
\citep{smith72, brown05, smith72a, smith72b}.

\subsection{Citations with Page Numbers}
\citet[p.~123]{smith72} says A, \citet[p.~234]{brown05} say B,
\citet[p.~345]{smith72a} say C, and \citet[p.~456]{smith72b} say D.
A equals B \citep[p.~123]{smith72}, B equals C \citep[p.~234]{brown05},
C equals D \citep[p.~345]{smith72a}, and D equals A \citep[p.~456]{smith72b}.


\section{Equations}
There are no instructions regarding equations.
\begin{equation}
y = a + X b
\end{equation}
where $a$ is a scalar,
$y$ and $b$ are vectors,
and $X$ is a matrix.
Of course, you can also use Greek symbols.
\begin{equation}
\theta = \alpha + \Psi \beta
\end{equation}
where $\alpha$ is a scalar,
$\theta$ and $\beta$ are vectors,
and $\Psi$ is a matrix.

\clearpage
\nocite{*}

\bibliography{apecon-ex}

\end{document}
